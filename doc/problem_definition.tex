% Riccardo Orizio
% 13 October 2015
% Problem definition 2.0

\documentclass[10pt, letterpaper]{article}
\usepackage[utf8]{inputenc}
\usepackage{mathtools}
\usepackage{amssymb}

% Every list item will start with a dash instead of a dot
\def\labelitemi{--}

\title{Problem definition}
\author{Riccardo Orizio}
\date{13 October 2015}

\begin{document}

\maketitle

\section{Problem definition}

Our problem concerns the need of a company to accomodate client requests to 
specific time slots where the company will have more profit, given the actual
requests status and the probable future requests.
This problem treats a small part of a bigger problem which is the problem of 
the attended home delivery services.

First of all, the company needs to choose if the new incoming request will be 
accepted or not: remember that every refused request is lost and cannot be 
recovered in the future, so every declined request will cost to the company 
because there will be no profit from it.
If a request is accepted, then it has to fit in the already known request 
calendar, knowing that every new request has a limited set of possible time 
windows in which it could be served (usually the time windows are given due
to geographical request studies) and that every request must respect its time 
window.

The goal of this problem is to bias the customers to make them choose what is 
more suitable for the company, so it could reach the maximum profit or it 
could serve the highest number of requests that it could do or whatever the 
objective of the company is.
The company will try to influence its customers through small price 
incentives\footnote{There is no need to place big incentives, small are good 
enough, knowledge given by home grocery delivery providers, Campbell and 
Savelsbergh 2006.} over the possible time windows listed to them, so there is 
a high probability that customers choose the time window that the company 
wants them to pick.
The problem is to place these incentives, that is to choose in which time 
window they have to be placed and how big every one of them has to be.

The problem is dynamic due to the fact that we keep receiving new requests even 
after our planning is started; it's stochastic because we try to predict future 
requests from the history and because we don't know when the request will come 
neither how big the demand of each one will be.
We use this information with the intention to keep the best incoming requests 
and discarding the worsts.

\section{Objective function}
The objective function could vary depending on the priorities of the company,
a couple examples are listed here:
\begin{itemize}
	\item \( MAX\ Profit = MAX\ \sum revenue -
								\sum delivery\_costs -
								\sum incentives \)

	\item \( MAX\ Orders\_delivered \)
\end{itemize}

\section{Variables}
\begin{description}
	\item[\(C\)] 
		customers spread into a limited geographical area, usually divided in
		sub-zones (e.g. ZIP-code division, districts)
	
	\item[\(V\)]
		company vehicles, each one having \(Q^{s}\) storage and \(Q^{f}\) 
		fuel capacity\footnote{Every vehicle could have different capacities.}

	\item[\(T\)]
		delivery time window\footnote{Time windows could be overlapped with 
		each other or not.} options, each being \(L\)\footnote{The larger 
		the better: we have more flexibility. (Campbell and Savelsbergh, 2006).}
		wide (time windows are chosen by the company itself)

	\item[\(I_{t}\)]
		possible incentives, one per time window

	\item[\(p_{i}^{t}\)]
		the probability that the customer \(i\) chooses the time window \(t\) 
		to get his items delivered
\end{description}

\section{Constraints}
\begin{itemize}
	\item 
		Don't exceed vehicles capacity, both storage and fuel (or delivery
		costs)

	\item 
		Make sure that every delivery will stay in the defined time-window,
		remembering also that every delivery has a service time\footnote{Time 
		between vehicle arrival to the customer and its departure.}

	\item
		\( \forall t \in T : I_{t} \geq 0 \)
\end{itemize}

\section{Assumptions}
\begin{itemize}
	\item 
		Real-time delivery are not available, we want at least 1-day notice
		to create the best plan that we can

	\item 
		Delivered products are not subject to any particular restriction:
		they are unbreakable and without expiry date (e.g.\ no food)

%	\item 
%		We treat every geographical region by itself, there isn't any possible
%		interaction between two or more regions

	\item 
		We keep a low number of \(C\) customers because the problem is hard
		to solve even with small sets of input

	\item
		\( \forall i\in V:Q_{i}^{s}=a,Q_{i}^{f}=b,a\,b\in\mathbb{{N}} \)
\end{itemize}

\section{Future extensions}
\begin{itemize}
	\item
		Everything that could vary from the assumptions

	\item
		Different time-windows: different \(L\), overlap between windows

	\item 
		Predict customers: keep track of every request done by every customer 
		so we could predict them even better, knowing their particular 
		time window preferences and using those for placing incentives only 
		when it's really necessary

	\item 
		Negative incentives: make the delivery cost more high for those 
		customers that demand a specific delivery that is unprofitable for 
		the company, without a negative incentive
\end{itemize}

\end{document}
