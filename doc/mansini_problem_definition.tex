%% LyX 2.1.4 created this file.  For more info, see http://www.lyx.org/.
%% Do not edit unless you really know what you are doing.
\documentclass[english]{article}
\usepackage[T1]{fontenc}
\usepackage[latin9]{inputenc}
\usepackage{amssymb}

\makeatletter
%%%%%%%%%%%%%%%%%%%%%%%%%%%%%% Textclass specific LaTeX commands.
\newenvironment{lyxlist}[1]
{\begin{list}{}
{\settowidth{\labelwidth}{#1}
 \setlength{\leftmargin}{\labelwidth}
 \addtolength{\leftmargin}{\labelsep}
 \renewcommand{\makelabel}[1]{##1\hfil}}}
{\end{list}}

\makeatother

\usepackage{babel}
\begin{document}

\part*{Vechicle Routing Problems with Time-Window Incentives}


\section*{The incentive problem}

This problem treats the home delivery service issue: the need for
a company to create a routing plan for its vehicles so they could
satisfy the highest number of customer orders keeping a good overall
profit.

After buying what he needs, a customer chooses a particular time-window
when he wants his purchase to be delivered. The company has to bias
customers, with some sort of incentives on the listed time-windows,
so there is a high probability that the customer chooses the time-window
that the company wants him to pick.

The main and most important part of this procedure is to place the
incentives, that is to choose in which time-window we should place
them and how big every incentive has to be.\\
\\
This problem could be described as follows, having:
\begin{lyxlist}{00.00.0000}
\item [{$C$}] customers/consumers spread into a limited geographical area
(e.g. ZIP-code division, districts)
\item [{$V$}] company vehicles, each one having $Q^{s}$ storage and $Q^{f}$
fuel capacity\footnote{Every vehicle could have a different capacities.}
\item [{$T$}] delivery time window\footnote{Time windows could be overlapped with each other or not.}
options, each being $L$\footnote{The larger the better: we have more flexibility. (Campbell and Savelsbergh,
2005)} wide (time windows are chosen by the company itself)
\item [{$I_{t}$}] possible incentives, one per time window
\item [{$p_{i}^{t}$}] the probability that the customer $i$ chooses the
time window $t$ to get his items delivered
\end{lyxlist}
Being restricted by:
\begin{itemize}
\item Don't exceed vehicles capacity, both storage and fuel (or delivery
costs)
\item Make sure that every delivery will stay in the defined time-window,
remembering also that every delivery has a service time\footnote{Time between vehicle arrival to the customer and its departure.}
\item $\forall t\in T:I_{t}\geq0$
\end{itemize}
Knowing that the final objective could be:
\begin{itemize}
\item $MAX\ Profit=MAX\ \sum revenue-\sum delivery\_costs-\sum incentives$
\item $MAX\ Orders\_delivered$\newpage{}
\end{itemize}
We also make some assumptions:
\begin{itemize}
\item Real-time delivery are not available, we want at least 1-day notice
to create the best plan that we can
\item Delivered products are not subject to any particular restriction:
they are unbreakable and without expiry date (e.g. no food)
\item We treat every geographical region by itself, there isn't any possible
interaction between two or more regions
\item We keep a low number of $C$ customers because the problem is hard
to solve even with small sets of input
\item $\forall i\in V:Q_{i}^{s}=a,Q_{i}^{f}=b,a\,b\in\mathbb{{N}}$
\end{itemize}
Future extensions or possible changes:
\begin{itemize}
\item Everything that could vary from the assumptions
\item Different time-windows: different $L$, overlap between windows
\item Predict customers: study customers behaviour to predict their choices,
knowing that they could be biased by incentives
\item Negative incentives: make the delivery cost more high for those customers
that demand a specific delivery that is unprofitable for the company,
without a negative incentive\end{itemize}

\end{document}
